%COMANDOS DE INFORMACION DE MATERIA
\newcommand{\codigoMateria}{86.05} %una "macro" para definir codigo de materia
\newcommand{\nombreMateria}{Analisis de Circuitos}
\newcommand{\descripcionTP}{Diseño y analisis de un filtro Notch}
\newcommand{\resumen}{
% TODO: hacer resumen de TP
hola
}
\newcommand{\tituloTP}{Trabajo práctico }
\newcommand{\facultad}{Facultad de Ingeniería}
\newcommand{\universidad}{Universidad de Buenos Aires}

%COMANDOS DE DATOS PERSONALES

\newcommand{\miNombre}{Vazquez, Rodrigo}
\newcommand{\miPadron}{98934}
\newcommand{\miMail}{rodrigomarianovazquez@gmail.com}
\newcommand{\fechaEntrega}{24 de Julio}

%MARGENES
\marginsize{2cm}{2cm}{1cm}{1.5cm} %izquierda, derecha, arriba, abajo

%HEADERS Y FOOTERS DEL INFORME
\pagestyle{fancy} % seleccionamos un estilo
\fancyhead{}
\fancyfoot{}
\lhead{\includegraphics[width=2.5cm]{imagenes/logofiuba.PNG}} % texto izquierda 
\rhead{\nombreMateria \, (\codigoMateria)} % texto centro de la cabecera
\cfoot{\thepage}

%OTRAS OPCIONES DE FORMATO
\newcommand{\HRule}{\rule{\linewidth}{1mm}} %linea negra de separacion
\date{} %saca la fecha
\raggedbottom %evita espacios en blancos grandes entre imagenes y textos

%PORTADA
\vspace*{-12mm}%esto, no entiendo como funciona pero hace lo que quiero (dejar un pequeño espacio)
\begin{center}
\fechaEntrega

\LARGE{\textsc{ \tituloTP }}\\[.3cm]
%\HRule \\[0.1cm]
\LARGE{\textbf{\descripcionTP}}\\[0.01cm]
\HRule\\[0.3cm]
\normalsize
\begin{tabbing}
	\begin{tabular}{ c c c }
		\miPadron & \miNombre & \miMail \\
	\end{tabular}
	\end{tabbing}
\end{center}
\HRule
\begin{abstract}
\justify
\resumen
\medskip
\end{abstract}
\justify
\HRule
\medskip