\usepackage[spanish]{babel}
\usepackage[utf8x]{inputenc}
\usepackage{ragged2e} % para justificar el texto
\usepackage{mathtools}
\usepackage{graphicx} % para insertar graficos/imagenes
\usepackage{anysize} % me permite definir los margenes como quiera
\usepackage{multirow} % para tablas con multicolumna
\usepackage{fancyhdr} % headers y footers
\usepackage{upgreek} % letras griegas sin cursiva
\usepackage{cancel}
\usepackage{siunitx}
\setlength{\headheight}{28pt}
\usepackage{subfig}
\usepackage{wrapfig}


% Paquetes para dibujar circuitos en LaTeX
\usepackage{circuitikz}
\usepackage{tikz}
\usepackage{float} %para que las figuras vayan donde las queremos.
 \addto\captionsspanish{
	\renewcommand{\tablename}{Tabla}
}

 
% \usepackage{amsmath}%modificación de numeración de ecuaciones
% \numberwithin{equation}{section} %tipo de numeración capitulo.ecuacion

%\usepackage[usenames]{color}

\renewcommand{\arraystretch}{1.2}%cambia el interlineado de las tablas

\usepackage{multicol,multirow}    % Permite poner multicolumnas mediantes:  \begin{multicols}{numero_columnas}   ...   \end{multicols}  